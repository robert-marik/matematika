\documentclass{article}
\usepackage[landscape, margin = 0.8cm, bottom=1.5cm, top=1.3cm]{geometry}
\usepackage[czech]{babel}
\usepackage[utf8]{inputenc}
\usepackage{amsmath, amsfonts}

\usepackage{multicol, enumerate, tikz}
\setlength{\columnseprule}{0.4pt}
\everymath{\displaystyle}

\def\dx{\,dx}
\def\dy{\,dy}
\def\dr{\,dr}
\def\dphi{\,d\phi}
\def\dA{\,dA}
\def\tg{\mathop{\mathrm{tg}}}
\def\arctg{\mathop{\mathrm{arctg}}}
\def\cotg{\mathop{\mathrm{cotg}}}

\pagestyle{empty}

\begin{document}


\abovedisplayskip 3 pt
\belowdisplayskip 3 pt
\let\phi\varphi
\footnotesize
\parindent 0 pt
\parskip 10 pt
\begin{multicols}{3}
\rightskip 0 pt plus 1 fill

{\centering \parskip 0 pt
\textbf{Cheatsheet Matematika}

(Robert Mařík, \today)

}

\textbf{Funkce}
\vspace*{-\baselineskip}
\begin{itemize}
\item Funkce jsou zobrazení mající na vstupu a výstupu reálná čísla. Používají se k vyjádření vzájemných relací zkoumaných veličin.
\item Podle  toho, zda zachovávají uspořádání vzorů i pro obrazy dělíme na rostoucí, klesající a ostatní (bez monotonie).
\item Podle toho, zda a jakým způsobem jsou symetrické dělíme na sudé, liché a ostatní (bez parity).
\item Nejjednoduššími funkcemi jsou přímá úměrnost (násobení konstantou, $y=kx$) a nepřímá úměrnost (násobení převrácené hodnoty konstantou, $y=\frac kx=k\frac 1x$).
\item Je-li $y=f(x)$, říkáme, že $y$ závisí na $x$. Veličina $y$ je závislá veličina, veličina $x$ je nezávislá veličina.
\end{itemize}

\textbf{Spojitost a limita (velmi zjednodušeně)}
\vspace*{-\baselineskip}
\begin{itemize}
\item Spojité funkce mají relativně malé změny funkčních hodnot při relativně malých změnách vstupních dat. Např. nejsou skoky.
\item Limita umožňuje doplnit funkční hodnotu funkce tak, aby se stala spojitou (odstranitelná nespojitost).
\end{itemize}

\textbf{Derivace}
\vspace*{-\baselineskip}

Dává do souvislosti změny závislé a nezávislé veličiny. 

$$\frac{\mathrm df}{\mathrm dx}:=\lim_{h\to 0}\frac{f(x+h)-f(x)}{h}$$

\begin{itemize}
\item Rychlost s jakou se mění $f$ při změnách $x$.
\item Změna $f$ vyvolaná jednotkovou změnou $x$.
\item Pro $f(t)$ je $\frac{\mathrm df}{\mathrm dt}$ rychlost růstu veličiny $f$ (v čase).
\item Jednotka derivace veličiny $f$ podle $x$ je stejná jako jednotka podílu těchto veličin.
\end{itemize}

\textbf{Konečné diference}
\vspace*{-\baselineskip}

Konečné diference slouží k numerickému aproximování derivace. Slouží také k odhadu derivace funkce dané tabulkou.

Dopředná diference: $\frac{\mathrm df}{\mathrm dx}=f'(x)\approx\frac{f(x+h)-f(x)}{h}$

Centrální diference: $\frac{\mathrm d f}{\mathrm dx}=f'(x)\approx  \frac{f(x+h)-f(x-h)}{2h}$

Druhá diference: $\frac{\mathrm d^2f}{\mathrm dx^2}=f''(x)\approx  \frac{f(x-h)-2f(x)+f(x+h)}{h^2}$

\columnbreak

\textbf{Aplikace derivace}

\begin{itemize}
\item Rovnice pro růst rychlostí úměrnou funkční hodnotě je
  $$\frac{\mathrm dx}{\mathrm dt}=kx.$$
  Rovnice pro růst rychlostí úměrnou vzdálenosti od stacionárního stavu je
  $$\frac{\mathrm dx}{\mathrm dt}=k(M-x).$$
  Řešením těchto modelů je možné získat časový průběh hledané funkce.
\item Lineární aproximace funkce $f$ v okolí bodu $x_0$ je
  $$f(x)\approx f(x_0)+ \frac{\mathrm df(x_0)}{\mathrm dx}(x-x_0).$$
\item Okamžitá rychlost. Derivace vystupuje všude tam, kde nestačí průměrná rychlost, ale požadujeme rychlost okamžitou. Je proto v matematické formulace naprosté většiny fyzikálních zákonů, pokud se nechceme specializovat pouze na děje probíhající konstantními rychlostmi (jako ve středoškolské fyzice).
\item Rovnice vedení tepla (pro parciální derivace) -- obsahuje časové i prostorové derivace, protože teplo se šíří v prostoru a teplota v daném místě se při vedení tepla mění s časem.
\item Řešení rovnice $f(x)=0$ je možno provést iteracemi
  $$x_{n+1}=x_n-\frac{f(x_n)}{f'(x_n)},$$
  kde $f'(x_n)=\frac{\mathrm df(x_n)}{\mathrm dx}$ je derivace v bodě $x_n$ a iterační proces začne v nepříliš špatném počátečním odhadu řešení $x_0$ (Newtonova metoda). Každý jeden krok metody zdvojnásobí počet desetinných míst při přibližném řešení rovnice.
\end{itemize}


\newcount\formulacount
\def\hopla{\advance\formulacount by 1 {\scriptsize ({\the\formulacount})\ }}
\def\derivace#1;#2\par{\kern 1pt\hopla $\displaystyle{(#1)'=#2}$\par\kern 2pt}
\def\integral#1;#2\par{\kern 1pt\hopla $\displaystyle{\int #1\dx=#2+c}$\par}

\textbf{Vzorce pro derivování}

{\parskip 0 pt

% \vspace*{-15pt}

\begin{multicols}2
  \derivace c;0

  \derivace x^n;n x^{n-1}

  \derivace e^x;e^x

  \derivace \ln x;\frac 1x

  \derivace \sin x;\cos x

  \derivace \cos x;-\sin x

  \derivace \tg x;\frac 1{\cos^2 x}

  \derivace \cotg x;-\frac 1{\sin^2 x}

  \derivace \arcsin x;\frac{1}{\sqrt {1-x^2}}

  \derivace \arccos x;-\frac{1}{\sqrt {1-x^2}}

  \derivace \arctg x;\frac 1{1+x^2}

\bigskip

\parskip 5 pt

1. $(u\pm v)'=u'\pm v'$

2. $(cu)'=cu'$

3. $(uv)'=u'v+uv'$

4. $\displaystyle{\Bigl(\frac {u}{v}\Bigr)'=\frac{u'v-uv'}{v^2}}$

5. $\displaystyle{\Bigl(u(v(x))\Bigr)'=u'(v(x))v'(x)}$

\end{multicols}

}


\textbf{Neurčitý integrál}

Neurčitý integrál je opak derivace, kdy z rychlosti změny určíme časový průběh měnící se veličiny. Je dána až na aditivní konstantu související s výchozím stavem.
Píšeme
$$\int f(x)\,\mathrm dx=F(x),$$
kde $F(x)$ je funkce splňující $\frac{\mathrm dF}{\mathrm dx}=f(x).$

\textbf{Vzorce pro integrování}

\vspace*{-8pt}
{
\parskip 0 pt
\hbox to \linewidth{
\begin{minipage}[t]{0.42\linewidth}
  \integral ;x

  \integral x^n; \frac{x^{n+1}}{n+1}

  \integral \frac 1x;\ln |x|

  \integral e^x;e^x

  \integral \sin x;-\cos x

  \integral \cos x;\sin x

\end{minipage}\hfil\vrule\hfil
\begin{minipage}[t]{0.53\linewidth}
\formulacount=6
\parskip 3pt

  \integral \frac 1{\cos^2 x};\tg x

  \integral \frac 1{\sin^2 x};-\cotg x


  \integral \frac1{\sqrt{1-x^2}};\arcsin x

  \integral \frac 1{1+x^2};\arctg x

  \integral \frac 1{1-x^2};\frac 1{2}\ln\left|\frac{1+x}{1-x}\right|

\end{minipage}}
}



\textbf{Určitý Newtonův integrál}

Newtonův integrál je opak derivace, kdy z rychlosti změny a z časového intervalu určíme změnu veličiny na daném intervalu. Na rozdíl od neurčitého integrálu je určitý integrál dán jednoznačně a není nutná informace o počátečním stavu. Počítáme vztahem 
$$\int_a^bf(x)\,\mathrm dx=F(b)-F(a),$$
kde $F(x)$ je funkce splňující $\frac{\mathrm dF}{\mathrm dx}=f(x).$

\textbf{Určitý Riemmannův integrál}

Riemmannův integrál umožňuje vypočítat součet nekonečné mnoha malých příspěvků k aditivní veličině. V praxi náhrada za nemožnost použít násobení v případě nekonstantních parametrů. Například dráha pohybu je rychlost násobená časem v případě pohybu konstantní rychlostí. V případě pohybu nekonstantní rychlostí je dráha místo součinu integrál rychlosti jako funkce času.

Riemmannův integrál se počítá stejně jako Newtonův. 

\vfill\null

\columnbreak


\textbf{Autonomní DR 1. řádu}
$$\frac{\mathrm dy}{\mathrm dt}=f(y)$$ 

Konstantních řešení je tolik, kolik je nulových bodů funkce $f(y)$.

Stabilní konstantní řešení jsou v bodech, kde je funkce $f$ klesající. 

Netabilní konstantní řešení jsou v bodech, kde je funkce $f$ rostoucí. 



% \vfill
% \null
% \penalty -500

% \columnbreak

\smallskip
\hrule


\textbf{Vektory}, $\vec a=(a_1,a_2,a_3)$, $\vec b=(b_1,b_2,b_3)$, 

Skalární součin vektorů
$\vec a\cdot\vec b=a_1 b_1 + a_2 b_2+a_3b_3$

Vektorový součin vektorů
\vspace*{-8pt}
$$\vec a\times \vec b=
\begin{vmatrix}
  \vec i & \vec j & \vec k\\
  a_1 & a_2 & a_3 \\
  b_1 & b_2 & b_3
\end{vmatrix}
$$

Délka vektoru  $|\vec a|=\sqrt{a_1^2+a_2^2+a_3^2}$

\textbf{Matice}

Součin matice a sloupcového vektoru je lineární kombinace sloupců matice, kdy koeficienty jsou komponentami vektoru.

$$
\begin{pmatrix}
  5& 2\\-3 & 1
\end{pmatrix}
\begin{pmatrix}
  22\\15
\end{pmatrix}
=
22
\begin{pmatrix}
  5\\-3
\end{pmatrix}
+
15
\begin{pmatrix}
  2\\1
\end{pmatrix}
=
\begin{pmatrix}
  140 \\-51
\end{pmatrix}
$$

\textbf{Determinanty}


Determinant 2$\times$2 \hfill$
\begin{vmatrix}
  a & b \\ c &d
\end{vmatrix}
=ad-bc
$

Determinant 3$\times$3 \\\null\hfill$
\begin{vmatrix}
  a & b & c \\ d & e &f \\ g & h & i
\end{vmatrix}
=aei-afh+bfg-bdi+cdh-ceg
$\hfill\null


\textbf{Matice rotace} pootočí bod nebo vektor o úhel $\theta$ proti směru hodinových ručiček. Inverze je poootočení o $-\theta$.

$$R(\theta)=
\begin{pmatrix}
  \cos\theta & -\sin \theta\\
  \sin\theta & \cos\theta
\end{pmatrix}$$

Zobrazení dané maticcí $A$ má v souřadnicích pootočených o $\theta$ proti směru hodinových ručiček matici $R(-\theta)AR(\theta)$.


\vfill\null

\columnbreak

\textbf{Vlastní čísla a vektory}

Vlastní vektory jsou vektory, které se zobrazí do stejného směru, poměr délek je vlastní číslo. Tj. vektor $\vec u$ je vlastním vektorem matice $A$ příslušným vlastnímu číslu $\lambda$, pokud platí $$A\vec u=\lambda\vec u.$$

Všechna vlastní čísla najdeme jako kořeny rovnice $$|A-\lambda I|=0.$$ Symetrická pozitivně definitní matice má tolik reálných vlastních čísel, kolik je její dimenze. 

Je-li $\lambda$ vlastní číslo, je vlastní vektor řešení rovnice $$(A-\lambda I)\vec u=0.$$

Vlastní směry pro dřevo odpovídají anatomickým směrům dřeva. V těchto směrech má odezva na podnět stejný směr jako podnět. Poměr velikosti odezvy a velikosti podnětu je roven vlastnímu číslu. Například pro difuzi vody je největší vlastní číslo v~podélném směru, protože v tomto směru vede dřevo vodu nejlépe.

\medskip\hrule\medskip

\textbf{Rovnice kontiuity} vyjadřuje celkovou bilanci stavové veličiny a nárůst množství způsobený přítomností zdrojů a změnami v toku přenášejícího tuto stavovou veličinu.


$${\frac{\partial u}{\partial t}=\sigma -\nabla\cdot \vec \jmath}$$

$u$ ... stavová veličina

$\frac{\partial u}{\partial t}$ ... rychlost růstu stavové veličiny

$\sigma$ ... vydatnost zdrojů stavové velčiny

$\vec\jmath$ ... tok stavové veličiny

$-\nabla\cdot \vec \jmath$ ... úbytek toku stavové veličiny (záporně vzatá divergence toku)



\textbf{Difuzní rovnice} je rovnice kontinuty doplněná předpokladem, že tok je úměrný záporně vzatému gradientu stavové veličiny.

$$\vec \jmath=-D\nabla \vec u$$

$${\frac{\partial u}{\partial t}=\sigma + \nabla\cdot \bigl(D\nabla u\bigr)}$$


\vfill
\columnbreak

\textbf{Difuzní rovnice v kartézských souřadnicích}


Budeme uvažovat diagonální difuzní koeficient. Například homogenní materiál, nebo otrotropní materiál ve kterém volíme osy v souladu s materiálovými vlastnostmi.

Obecný tvar difuzní rovnice v kartézských souřadnicích je
$$\frac{\partial u}{\partial t}=\sigma +
\frac{\partial }{\partial x}\left(D_x \frac{\partial u}{\partial x}\right)
+
\frac{\partial }{\partial y}\left(D_y \frac{\partial u}{\partial y}\right)
+
\frac{\partial }{\partial z}\left(D_z \frac{\partial u}{\partial z}\right).
$$

Ve \textit{stacionárním případě} (sledujeme stav po dosažení rovnováhy) položíme navíc $\frac{\partial u}{\partial t}=0$.

Pro \textit{bezzdrojový případ} (stavová veličina nemůže vznikat ani zanikat) položíme navíc $\sigma=0$.

Studujeme-li materiál současně \textit{homogenní} a s \textit{lineárními materiálovými vlastnostmi} (v všech místech má materiál stejné vlastnosti a ty se nemění při změně stavové veličiny), potom kvaziderivace
$$\frac{\partial }{\partial x}\left(D_x \frac{\partial u}{\partial x}\right)$$
upravíme a použijeme druhé derivace
$$D_x \frac{\partial^2 u}{\partial x^2}$$ (a pro ostatní proměnné analogicky).

U \textit{izotropního materiálu} (má ve všech směrech stejné vlastnosti) nerozlišujeme materiálové charakteristiky v jednolivých směrech, klademe $D_x=D_y=D_z=D.$

Pro méně dimenzí  použijeme pouze odpovídající počet difuzních členů. Jednotlivé předpoklady je možno libovolně kombinovat.


\end{multicols}
\end{document}


