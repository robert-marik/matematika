\documentclass[11pt]{amsart}

\usepackage[utf8]{inputenc}
\usepackage{amsmath}
\usepackage[czech]{babel}
\usepackage[a4paper, margin=.75in]{geometry}
\begin{document}


\title{Integrál}
\maketitle
\thispagestyle{empty}


Integrál je v jistém smyslu nástroj opačný k derivaci. Zatímco derivace umožní z časového průběhu veličiny
získat informaci o její rychlosti změny, integrál umožní z rychlosti změny získat informaci o~celkové změně veličiny
za dané časové období (určitý integrál) nebo o~časovém přůběhu veličiny na základě její rychlosti změny (neurčitý integrál, nebo integrál jako funkce horní meze).

\section*{Neurčitý integrál, primitivní funkce}

Primitivní funkce k funkci $f(x)$ je funkce $F(x)$, jejíž derivace je rovna funkci $f(x)$, tj. 
\[\frac{\mathrm dF}{\mathrm dx} = f(x).\] Primitivní funkce existuje ke každé spojité funkci a je určena jednoznačně až na aditivní konstantu.
Neurčitý integrál funkce $f(x)$ je množina všech primitivních funkcí k funkci $f(x)$ a značí se
\[\int f(x) \,\mathrm dx.\]

\section*{Interpretace a použití neurčitého integrálu}

Uvažujme proces, kde se sledovaná veličina mění známou rychlostí $f(t)$ v závislosti na čase $t$.
Neurčitý integrál funkce $f(t)$ nám umožní určit časový průběh sledované veličiny $F(t)$ na základě znalosti její rychlosti změny $f(t)$.

\begin{itemize}
  \item 
Pokud funkce $f(t)$ udává rychlost pohybu tělesa v čase $t$, pak primitivní funkce $F(t)$ k funkci $f(t)$ udává polohu tělesa v čase $t$, přičemž aditivní konstanta určuje počáteční polohu tělesa v~čase $t=0$.
\item
Pokud funkce $f(t)$ udává rychlost růstu teploty v čase $t$, pak primitivní funkce $F(t)$ k funkci $f(t)$ udává teplotu v čase $t$, přičemž aditivní konstanta určuje počáteční teplotu v~ čase $t=0$.
\item
Pokud funkce $f(t)$ udává rychlost s jakou se mění množství vlhkosti v materiálu v závislosti  
na čase $t$, pak primitivní funkce $F(t)$ k funkci $f(t)$ udává množství vlhkosti v materiálu v čase $t$, přičemž aditivní konstanta určuje počáteční množství vlhkosti v čase $t=0$.
\end{itemize}

\section*{Určitý integrál}

Určitý integrál funkce $f(x)$ na intervalu $[a, b]$ je definován jako nárůst primitivní funkce 
k~funkci $f(x)$ mezi body $a$ a $b$, tj. 
\[\int_a^b f(x) \,\mathrm dx = F(b) - F(a),\] kde $F(x)$ je libovolná primitivní funkce k funkci $f(x)$.

\section*{Interpretace a použití určitého integrálu}

Geometricky určitý integrál představuje obsah plochy mezi křivkou funkce $f(x)$ a osou $x$ na intervalu $[a, b]$. Pokud funkce $f(t)$ udává rychlost růstu sledované veličiny v čase, pak určitý integrál funkce $f(t)$ na intervalu $[a,b]$ 
\[\int_a^b f(t) \,\mathrm dt\]
udává celkovou změnu sledované veličiny v čase od okamžiku $a$ do okamžiku $b$. 
Fyzikálně určitý integrál vystupuje také v aplikacích, kde 
je určována hodnota aditivní veličiny na základě příspěvků k celkové hodnotě této veličiny.




\vfill
{\bfseries Instrukce:} \itshape V textu jsou typografické chyby týkající se sazby matematických výrazů. 
Najděte je a rukou vyznačte barevně ve vytištěném textu. Papír podepište v pravém horním rohu, napište zkratku 
oboru, který studujete
a odevzdejte na přednášce. 13. listopadu Tam proběhne i zpětná vazba. Protože už nejsou hlavní cvičení, 
nebude možné práci vhazovat
do schránky. 

\end{document}
