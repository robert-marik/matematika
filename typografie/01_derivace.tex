\documentclass[11pt]{amsart}

\usepackage[utf8]{inputenc}
\usepackage{amsmath}
\usepackage[czech]{babel}
\usepackage[a4paper, margin=.75in]{geometry}
\begin{document}


\title{Derivace}
\maketitle
\thispagestyle{empty}

Derivace funkce je důležitý pojem při matematickém modelování
přírodních dějů a jevů. Popisuje jak rychle se mění hodnota funkce
vzhledem k její nezávislé proměnné. V praxi se nejčastěji setkáváme s
funkcemi vyjadřujícími závislost nějaké veličiny na čase nebo prostorové
souřadnici.

\section*{Definice}
Derivace funkce f(x) v bodě x je definována jako limita
\[
\frac{\mathrm df}{\mathrm dx} = \lim_{h \to 0} \frac{f(x + h) - f(x)}{h}.
\]
Zlomek v této definici vyjadřuje průměrnou rychlost změny funkce $f$ na intervalu od x do x + h. Limitní přechod 
v definici derivace slouží k tomu, aby se délka tohoto intervalu zmenšila na nulu a interval zdegeneroval na bod. Tím získáme okamžitou rychlost změny funkce v bodě $x$.

\section*{Interpretace a použití}
Nejčastěji se v aplikacích setkáme s derivací podle času. Obvyklá slovní interpretace derivace je \textit{rychlost změny} nebo 
ekvivalentně \textit{změna za jednotku času}. Například derivace teploty $T$ horké kávy podle času t může mít hodnotu 
\[\frac{\mathrm dT}{\mathrm dt}=-4^\circ  C/{min}\]
To znamená, že teplota kávy klesá rychlostí 4 stupně Celsia za minutu.

Derivace je základním nástrojem pro popis přírodních dějů, protože vnější podmínky definují, jak rychle se mění parametry systému
v čase. Například rychlost ochlazování kávy (tj. rychlost poklesu teploty kávy) je úměrná rozdílu teploty kávy a teploty okolí.
Skutečnost, že v popisu tohoto procesu figuruje slovo \textit{rychlost}, naznačuje, že se při modelování tohoto děje pracuje 
s derivací podle času. Matematický model poté má tvar
\[
\frac{\mathrm dT}{\mathrm dt} = -k*(T - T_{M})
\]

kde $T_M$ je teplota okolí a k je konstanta závislá na vlastnostech systému (např. na materiálu šálku, 
jeho tvaru, velikosti povrchu atd.).

\section*{Lineární aproximace}
Derivaci je možné využít k lineární aproximaci funkčních vztahů pomocí vzorce 
\[
f(x + h) \approx f(x) + \frac{\mathrm df}{\mathrm dx}*h.
\]
Klasickou ukázkou je skutečnost, že mnoho konstitučních vztahů má tvar přímé úměrnosti: 
Darcyho zákon, Hookův zákon, Fourierův zákon, Fickův zákon atd.

Linearizaci je možné též využít při řešení rovnic Newtonovou metodou. V tomto případě se nulový bod 
funkce hledá jako nulový bod její lineární aproximace v okolí aktuálního odhadu řešení. Výsledkem
je zpřesnění původního odhadu řešení, kdy se počet platných číslic přibližně zdvojnásobí s každou iterací.

\section*{Parciální derivace}

V případě funkcí více proměnných je možné definovat parciální derivace podle jednotlivých proměnných. 
Například při vedení tepla ve stěně je teplota funkcí prostorových souřadnic a času. Poté rychlost 
růstu teploty jako funkce času a rychlost růstu teploty jako funkce prostorové souřadnice jsou 
\[
\frac{\partial T}{\partial t} \quad \mathrm{a} \quad \frac{\partial T}{\partial x}.
\]

\vfill
{\bfseries Instrukce:} \itshape V textu jsou typografické chyby týkající se sazby matematických výrazů. 
Najděte je a vyznačte barevně ve vytištěném textu. Papír podepište v pravém horním rohu, napište zkratku oboru, který studujete
a odevzdejte na příští přednášce. Datum je u zadání úkolu v Moodle.

\end{document}