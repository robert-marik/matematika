\documentclass[11pt, reqno]{amsart}
%DIF LATEXDIFF DIFFERENCE FILE
%DIF DEL 04_difuzni_rovnice.tex         Mon Jan  5 09:49:41 2026
%DIF ADD 04_difuzni_rovnice_fixed.tex   Mon Jan  5 09:51:22 2026

\usepackage[utf8]{inputenc}
\usepackage[czech]{babel}
\usepackage{amsmath}
\usepackage[a4paper, margin=.75in]{geometry}
%DIF PREAMBLE EXTENSION ADDED BY LATEXDIFF
%DIF UNDERLINE PREAMBLE %DIF PREAMBLE
\RequirePackage[normalem]{ulem} %DIF PREAMBLE
\RequirePackage{color}\definecolor{RED}{rgb}{1,0,0}\definecolor{BLUE}{rgb}{0,0,1} %DIF PREAMBLE
\providecommand{\DIFadd}[1]{{\protect\color{blue}\uwave{#1}}} %DIF PREAMBLE
\providecommand{\DIFdel}[1]{{\protect\color{red}\sout{#1}}}                      %DIF PREAMBLE
%DIF SAFE PREAMBLE %DIF PREAMBLE
\providecommand{\DIFaddbegin}{} %DIF PREAMBLE
\providecommand{\DIFaddend}{} %DIF PREAMBLE
\providecommand{\DIFdelbegin}{} %DIF PREAMBLE
\providecommand{\DIFdelend}{} %DIF PREAMBLE
\providecommand{\DIFmodbegin}{} %DIF PREAMBLE
\providecommand{\DIFmodend}{} %DIF PREAMBLE
%DIF FLOATSAFE PREAMBLE %DIF PREAMBLE
\providecommand{\DIFaddFL}[1]{\DIFadd{#1}} %DIF PREAMBLE
\providecommand{\DIFdelFL}[1]{\DIFdel{#1}} %DIF PREAMBLE
\providecommand{\DIFaddbeginFL}{} %DIF PREAMBLE
\providecommand{\DIFaddendFL}{} %DIF PREAMBLE
\providecommand{\DIFdelbeginFL}{} %DIF PREAMBLE
\providecommand{\DIFdelendFL}{} %DIF PREAMBLE
%DIF COLORLISTINGS PREAMBLE %DIF PREAMBLE
\RequirePackage{listings} %DIF PREAMBLE
\RequirePackage{color} %DIF PREAMBLE
\lstdefinelanguage{DIFcode}{ %DIF PREAMBLE
%DIF DIFCODE_UNDERLINE %DIF PREAMBLE
  moredelim=[il][\color{red}\sout]{\%DIF\ <\ }, %DIF PREAMBLE
  moredelim=[il][\color{blue}\uwave]{\%DIF\ >\ } %DIF PREAMBLE
} %DIF PREAMBLE
\lstdefinestyle{DIFverbatimstyle}{ %DIF PREAMBLE
	language=DIFcode, %DIF PREAMBLE
	basicstyle=\ttfamily, %DIF PREAMBLE
	columns=fullflexible, %DIF PREAMBLE
	keepspaces=true %DIF PREAMBLE
} %DIF PREAMBLE
\lstnewenvironment{DIFverbatim}{\lstset{style=DIFverbatimstyle}}{} %DIF PREAMBLE
\lstnewenvironment{DIFverbatim*}{\lstset{style=DIFverbatimstyle,showspaces=true}}{} %DIF PREAMBLE
%DIF END PREAMBLE EXTENSION ADDED BY LATEXDIFF

\begin{document}


\title{Rovnice kontiuity, rovnice difuze}
\maketitle
\thispagestyle{empty}

Rovnice kontinuity a rovnice difuze jsou základními nástroji pro 
popis transportních jevů v přírodě.

\section{Rovnice kontinuity}

Rovnice kontinuity vyjadřuje zákon zachování nějaké veličiny (hmotnosti či jiné míry množství, 
energie, apod.) v čase při transportu této veličiny prostorem. Tato veličina definuje
stav systému a proto se nazývá \textit{stavová veličina}.
Obecně lze rovnici kontinuity zapsat ve tvaru
\[
\frac{\partial u}{\partial t} = \sigma - \nabla \cdot \vec{\jmath},
\]
\DIFdelbegin \DIFdel{\hspace*{\parindent}
kde u }\DIFdelend \DIFaddbegin \DIFadd{kde $u$ }\DIFaddend je hustota stavové veličiny, $\sigma$ je zdrojová 
funkce a \(\vec{\jmath}\) je hustota toku stavové veličiny.

Člen na levé straně rovnice představuje změnu stavové veličiny za jednotku času.
Pravá strana vysvětluje, že k této změně dochází generováním veličiny
ve zdrojích a změnou v toku stavové veličiny prostorem.
První člen na pravé straně rovnice představuje přírůstek stavové veličiny za jednotku času 
vygenerovaný zdroji (kladná hodnota) nebo ztracený ve spotřebičích (záporná hodnota).
Druhý člen na pravé straně rovnice představuje (včetně znaménka) zeslabení toku
stavové veličiny v důsledku toku prostorem. Záporné znaménko před operátorem divergence
zaručuje, že pokud je tok stavové veličiny z prostoru ven, pak tato
stavová veličina v prostoru ubývá.

\section{Rovnice difuze}

V přírodě je tok stavové veličiny zpravidla vyvolán tím, že rozložení 
stavové veličiny v prostoru je nerovnoměrné a příroda se snaží 
prouděním tuto nerovnováhu vyrovnat. Například pokud má v tepelně 
vodivém tělese jedna část vyšší teplotu než druhá, pak teplo
v tělese proudí z teplejší části do chladnější. Podobně, 
pokud je v jednom místě tělesa vyšší koncentrace nějaké látky než v jiném místě,
pak tato látka v tělese difunduje z místa s vyšší koncentrací do místa s nižší koncentrací.

Tento jev lze matematicky popsat pomocí tzv. Fickova zákona, který říká, že
tok stavové veličiny je úměrný záporně vzatému gradientu hustoty stavové veličiny, tedy
\[\vec{\jmath} = - D \nabla u,
\]
\DIFdelbegin \DIFdel{\hspace*{\parindent}}\DIFdelend kde \(D\) je koeficient difuze. Tento zákon říká, že tok stavové veličiny směřuje 
od míst s vyšší hustotou stavové veličiny k místům s nižší hustotou. V případě
anizotropního prostředí nejsou nutně vektory $\vec\jmath$ a $\nabla u$ rovnoběžné
a proto má koeficient \DIFdelbegin \DIFdel{D }\DIFdelend \DIFaddbegin \DIFadd{$D$ }\DIFaddend maticový charakter.

Dosazením Fickova zákona do rovnice kontinuity dostaneme rovnici difuze ve tvaru
\[\frac{\partial u}{\partial t} = \sigma + \nabla \cdot (D \nabla u).
\]
Tato rovnice popisuje, jak se hustota stavové veličiny mění
v čase v důsledku přítomnosti zdrojů a difuzního transportu stavové veličiny prostorem.
S rovnicí difuze se často setkáváme v zápise pomocí souřadnic. Zvolíme-li souřadnou soustavu tak, 
že osy jsou ve vlastních směrech matice \(D\), pak lze rovnici difuze zapsat ve tvaru
\[\frac{\partial u}{\partial t} = \sigma + 
\frac{\partial}{\partial x}\left(D_x \frac{\partial u}{\partial x}\right) + 
\frac{\partial}{\partial y}\left(D_y \frac{\partial u}{\partial y}\right) 
\]
\DIFdelbegin \DIFdel{\hspace*{\parindent}}\DIFdelend v obecném případě a ve tvaru 
\[\frac{\partial u}{\partial t} = \sigma + 
D_x \frac{\partial^2 u}{\partial x^2} + 
D_y \frac{\partial^2 u}{\partial y^2}
\]
\DIFdelbegin \DIFdel{\hspace*{\parindent}}\DIFdelend v případě, že koeficienty difuze \DIFdelbegin \DIFdel{\(Dx\) a \(Dy\) }\DIFdelend \DIFaddbegin \DIFadd{\(D_x\) a \(D_y\) }\DIFaddend jsou konstantní. To nastane, pokud 
je prostředí homogenní a materiálové vlastnosti nezávisí na stavové veličině \(u\)\DIFdelbegin \DIFdel{.
}%DIFDELCMD < 

%DIFDELCMD < \vfill
%DIFDELCMD < {\bfseries %%%
\DIFdel{Instrukce:}%DIFDELCMD < } \itshape %%%
\DIFdel{V textu jsou typografické chyby týkající se sazby matematických výrazů. 
Najděte je a rukou vyznačte barevně ve vytištěném textu. Papír podepište v pravém horním rohu, napište zkratku 
oboru, který studujete
a odevzdejte na přednášce 18. prosince. Tam proběhne i zpětná vazba. Protože už nejsou hlavní cvičení, 
nebude možné práci vhazovat
do schránky. Kombinovaní studenti mohou zaslat emailem v PDF formátu}\DIFdelend .



\end{document}
