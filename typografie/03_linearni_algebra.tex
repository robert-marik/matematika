\documentclass[11pt]{amsart}

\usepackage[utf8]{inputenc}
\usepackage{amsmath}
\usepackage[czech]{babel}
\usepackage[a4paper, margin=.75in]{geometry}
\begin{document}


\title{Lineární algebra}
\maketitle
\thispagestyle{empty}

\section*{Vektory}

\textit{Vektor} je uspořádaná množina čísel, která může reprezentovat bod v prostoru,
směr nebo jiný matematický objekt. V praxi reprezentuje často 
hnací faktor transportního jevu, například spád teploty pro vedení tepla, spád 
koncentrace pro difuzi nebo spád hydraulické výšky pro proudění podzemní vody.

Základní operací s vektory je lineární kombinace vektorů, která je definována
tak, že předepsaná operace se provádí po složkách. 

\section*{Matice}

\textit{Matice} reprezentuje lineární zobrazení mezi dvěma vektorovými prostory. Například v případě
vedení tepla ve stěně matice popisuje, jaký tok tepla je vyvolán určitý spádem
teploty.
Součin matice $A$ a sloupcového vektoru $\vec u$ je vektor, který vznikne jako
lineární kombinace sloupců matice $A$ s koeficienty z vektoru $\vec u$, tj. 
\[
A \vec u = u_1 \vec a_1 + u_2 \vec a_2 + ... + u_n \vec a_n
\]
, kde $\vec a_1, \vec a_2, ..., \vec a_n$ jsou sloupce matice $A$ a $u_1, u_2, ..., u_n$ jsou složky vektoru $\vec u$.

Z praktického hlediska mají důležité postavení vektory, které se transformují na vektory rovnoběžné se svým
vzorem, tj. vektory, pro které platí
\[A \vec v = \lambda \vec v
\]
, kde $\lambda$ je reálné číslo. V tomto případě se vektor v
nazývá
\textit{vlastní vektor} matice A. 
U materiálů s anizotropními vlastnostmi (např. kompozity, horniny, dřevo)
vlastní vektory určují směry, ve kterých materiál vykazuje chování, které je
současně extrémní (např. maximální nebo minimální vodivost) a jednoduché
(odezva na podnět je rovnoběžná s tímto podnětem). Vlastní hodnoty pak udávají 
míru této odezvy. Pokud jsou vlastní směry materiálu navzájem ortogonální, lze je využít
k vytvoření zjednodušeného matematického modelu, který popisuje chování materiálu
v libovolném směru jako kombinaci chování v těchto vlastních směrech.
Matematickým důsledkem je, že matici materiálových vlastností je možné převést
na diagonální tvar.

Součin dvou matic $A$ a $B$ je definován jako matice, jejíž sloupce jsou získány
jako součiny matice $A$ a jednotlivých sloupců matice $B$. Neutrálním prvkem
maticového součinu je jednotková matice $I$, která má na hlavní diagonále
jedničky a jinde nuly. Pokud pro matici $A$ existuje matice $A^{-1}$ taková, že platí
\[A A^{-1} = A^{-1} A = I\]
potom se matice $A^{-1}$ nazývá \textit{inverzní matice} k matici $A$.

\section*{Soustavy lineárních rovnic}

Pomocí maticového součinu je možné zapsat \textit{soustavy lineárních rovnic}
ve tvaru
\[A X = B,\]
{}\indent kde $A$ je matice koeficientů, X je vektor neznámých a B je vektor pravých stran rovnic.
Řešení soustavy je možné získat pomocí inverzní matice jako
\[X = A^{-1} B\]
Tento postup je možný v případě, že matice A je čtvercová a má nenulový
determinant ($det A\neq 0$). Výpočet inverzní matice je však výpočetně náročný a v praxi se používají
efektivnější metody řešení soustav lineárních rovnic. Příkladem je \textit{Jordanova
metoda}, která je založena na faktu, že je snadné získat inverzní matici k
diagonální matici. Je-li matice A součtem diagonální matice D a matice T, 
kde matice T má na hlavní diagonále nuly, pak je možné rovnici přepsat do tvaru 
\[X = D^{-1}(B-TX)\]
a řešit pomocí iteračního vztahu 
\[X_{k+1} = D^{-1}(B-TX_k).\]





\vfill
{\bfseries Instrukce:} \itshape V textu jsou typografické chyby týkající se sazby matematických výrazů. 
Najděte je a rukou vyznačte barevně ve vytištěném textu. Papír podepište v pravém horním rohu, napište zkratku 
oboru, který studujete
a odevzdejte na přednášce 11. prosince. Tam proběhne i zpětná vazba. Protože už nejsou hlavní cvičení, 
nebude možné práci vhazovat
do schránky. 


\end{document}